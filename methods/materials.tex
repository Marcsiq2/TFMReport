\chapter{Materials}
\label{chap:materials}
For this project, we have used several materials which can be divided into 3 categories: hardware, software and data.

\section*{Hardware}
\begin{itemize}[noitemsep]
\item Roland GK-3: the hexaphonic recordings were done using this special divided pick-up
\item Breakout Box~\cite{OGrady2009}: this adaptor box was needed in order to convert the output from the GK-3 to 6 standard Jack connectors.
\item PC: Intel Core i5-6600 CPU @ 3.30GHz, 16.0GB RAM
\end{itemize}

\section*{Software}
\begin{itemize}[noitemsep]
\item ProTools HD 10: it was used in order to generate a mix the 6 strings channels. It also was used to synthesize midi both from the transcribed performances and from the predicted performance.
\item MuseScore 2: the scores in XML format for the performances were written using MuseScore.
\item Python: the code for transcribing guitar performances was developed by using Python.
\item Essentia~\cite{bogdanov2013essentia}: we used a few algorithms (in order to transcribe guitar) from this open-source C++ library for audio analysis and audio-based music information retrieval.
\item Matlab: the code for extracting the features from the performance and from the scores was developed by using Matlab.
\item MIDI Toolbox~\cite{eerola2004midi}: the MidiToolBox Library (implemented in Matlab) allowed us to process easily MIDI data.
\item Weka Data Mining Software~\cite{hall2009weka}: it was used to train and test different machine learning models, to implement feature selection and to analyze the results.
\end{itemize}

\section*{Data}
For this work we used a set of three recordings done by Helena Bantula for her Master's thesis consisting of one recording of \textit{Darn that dream} by Jimmy Van Heusen and Eddie De.Lange and two recordings of \textit{Suite en la} by Manuel M. Ponce.

Their corresponging scores where written using Musescore and extracted as XML files.

\cleardoublepage