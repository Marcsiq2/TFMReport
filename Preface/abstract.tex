\begin{abstract}
\todo[inline]{Estructura del abstract: \\1.Presentación del problema \\2.Previous work \\3.However... \\4.What you propose \\5.Major findings}

In this Master's Thesis, we present a work on a machine learning approach to automatically generate expressive performances from non expressive music scores of polyphonic guitar. We are going to treat guitar as an hexaphonic instrument and capture each string separately in order to obtain a better transcription. Features extracted from the scores and the corresponding audio recordings performed by a professional guitarist are going to be used to train computational models for guitar modelling and predicting performance actions such as onset deviation, duration deviation or energy ratio.

\begin{center}{
    \large\bf Resum}
  \end{center}
 
En aquesta tesi de màster treballarem en la generació automàtica de models d'expressivitat a partir de partitures musicals no expresives per a guitarra polifònica utilitzant aprenentatge automàtic. Tractarem la guitar com un instrument hexafònic i capturarem cada corda per separat, amb la finalitat d'obtenir una millor transcripció. Les característiques extretes a partir de les partitures i les corresponents gravacions d'àudio realitzades per un guitarrista professional seran utilitzades per entrenar models computacionals per a modelar i predir les petites accions expressives o ornamentacions de la melodia com deviació en onsets, en duració o en energia.

\end{abstract}

