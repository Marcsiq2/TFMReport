\begin{abstract}

% In this Master's Thesis, we present a work on a machine learning approach to automatically generate expressive performances from non expressive music scores of polyphonic guitar. We are going to treat guitar as an hexaphonic instrument and capture each string separately in order to obtain a better transcription. Features extracted from the scores and the corresponding audio recordings performed by a professional guitarist are going to be used to train computational models for guitar modelling and predicting performance actions such as onset deviation, duration deviation or energy ratio.

Computational modelling of expressive music performance has been widely studied in the past. While previous work in this area has been mainly focused on classical piano music, there has been very little work on guitar music, and such work has focused on monophonic guitar playing. In this work, we present a machine learning approach to automatically generate expressive performances from non expressive music scores for polyphonic guitar. We treated guitar as an hexaphonic instrument, obtaining a polyphonic transcription of performed musical pieces. 
Features were extracted from the scores and performance actions were calculated from the deviations of the score and the performance. Machine learning techniques were used to train computational models to predict the aforementioned performance actions. Qualitative and quantitative evaluations of the models and the predicted pieces were performed. 

\newpage
\cleardoublepage

\begin{center}{
    \large\bf Resum}
  \end{center}
 %\todo[inline]{Traduir abstract al català}
 
% En aquesta tesi de màster treballarem en la generació automàtica de models d'expressivitat a partir de partitures musicals no expresives per a guitarra polifònica utilitzant aprenentatge automàtic. Tractarem la guitar com un instrument hexafònic i capturarem cada corda per separat, amb la finalitat d'obtenir una millor transcripció. Les característiques extretes a partir de les partitures i les corresponents gravacions d'àudio realitzades per un guitarrista professional seran utilitzades per entrenar models computacionals per a modelar i predir les petites accions expressives o ornamentacions de la melodia com deviació en onsets, en duració o en energia.

El modelatge computacional de interpretacions expressives de peces músicals ha estat àmpliament estudiat en el passat. Tot i que el treball previ en aquesta àrea s'ha centrat principalment en la música clàssica per piano, hi ha hagut molt poc treball sobre música per guitarra i aquest s'ha centrat en la guitarra monofònica. En aquest treball, utilitzem aprenentatge automàtic per generar automàticament interpretacions expressives a partir de partitures de música no expressiva per a guitarra polifònica. Tractem la guitarra com a un instrument hexafònic, obtenint una transcripció polifònica de les peces musicals interpretades.
A partir de les partitures s'han extret diverses característiques i s'han calculat accions interpretatives a partir de les desviacions entre la partitura i la interpretació del músic. Diverses tècniques d'aprenentatge automàtic s'han utilitzat per entrenar models computacionals i predir les accions interpretatives esmentades anteriorment. Finamlent s'han realitzat avaluacions qualitatives i quantitatives dels models i les peces predites.

\end{abstract}

